\documentclass[a4paper]{article}
\usepackage[utf8]{inputenc}
\usepackage{amsmath}
\usepackage{amssymb}
\title{SDE Simulation and Statistic}

\date{Summarised: DATE}
\author{M Sidik Augi Rahmat}
\begin{document}
\maketitle


\section{Brownian Motion}
A stochastic process is a t-parametric system of random variables $\{ X(t), t \in I\}$ where I is and interval of reals or a discrete set of indices. Or as its easy
stochastic process is when random variable depends on some parameter, it could be time or another. One of example of stochastic process is Brownian motion. A Brownian
motio $\{ X(t), t\geq 0 \}$ is a t-parametric system of random variables, for which satisfies the following properties:
\begin{itemize}
    \item $X_0 = 0$
    \item The Increment are stationary and independent.
    \item It is a martingale. (Martingale is pretty hard because need to know probability, measusere space and stuff. But basically martingale is a concept where
    we double the bet if we lose in gambling. It because the random variable is independent each other.)
    \item It has continuos path, but nowhere differentiable
    \item $X_t - X_s \approx \mathcal{N}(0, t - s)$ for $t \geq s \geq 0$. It means  at the time $t>s$ the variable $X$ has a normal probability distribution.
\end{itemize}
A Brownian motion with parameters $\mu = 0$, $\sigma^2 = 1$ is called Wiener process.
In our simulation, each increment is such that:
\begin{equation}
    X_{t_{i} + \Delta t} - X_{t_{i}} = \Delta X_i \approx \mathcal{N}(\mu \Delta t, \sigma^2 \Delta t)
\end{equation}
The process at time $T$ is given by $X_T = \Sigma_i \Delta X_i$ and follows the distribution:
\begin{equation}
    X_T \approx \mathcal{N}(\mu T, \sigma^2 T)
\end{equation}

A Brownian motion $\{ X(t), t\geq 0\}$ can be characterized by its deterministic and fluctuatuing components. Its Increments $dX(t)$ can be expressed in the following form 
of total differential 
\begin{equation}
    dX(t) =\mu dt + \sigma dW(t)
    \label{eq:SDE}
\end{equation}
where $dW(t)$ is Wiener process. Equation \ref{eq:SDE} is called stochastic differential equation.


\section{Geometric Brownian Motion}
If $\{ X(t), t\geq 0\}$ is a Brownian motion with parameters $\mu, \sigma$ and $y_o \in \mathbb{R} ^ +$, then the system of random variables $\{ Y(t), t\geq 0\}$
\begin{equation}
    Y(t) = y_0 e^{X(t)}, t \geq 0
\end{equation}
is called geometric Brownian motion. Expectef value and variance for geometric Brownian motion is defined basically
\begin{equation}
    E(Y(t)) = y_0 e^{ut + \frac{\sigma ^ 2 t}{2}}, \quad Var(Y(t)) = y_0 ^ {2} e ^ {2 \mu t + \sigma^2 t} ( e ^ {\sigma ^ 2 t} - 1)
\end{equation}

adadadaad


\end{document}
